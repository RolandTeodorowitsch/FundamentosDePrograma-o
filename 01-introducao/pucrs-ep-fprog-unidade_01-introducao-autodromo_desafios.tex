\documentclass[aspectratio=169]{beamer}
\usepackage[utf8]{inputenc}
\usepackage[T1]{fontenc}
\usepackage[brazil]{babel}
\usepackage{graphics,amssymb,amsfonts,amsmath}
\usepackage{tikz}
\usepackage{enumerate,hyperref}
\usepackage{palatino}
\usepackage{ragged2e}
\usepackage{minted}
\usepackage{booktabs}
\usetheme{AnnArbor}
\usecolortheme{orchid}
\usefonttheme[onlymath]{serif}

\newminted{java}{bgcolor=cyan!10}

\title[\sc{Introdução (Autódromo)}]{Introdução (Autódromo)}
\author[Roland Teodorowitsch]{Roland Teodorowitsch}
\institute[FPROG - EP - PUCRS]{Fundamentos de Programação - Escola Politécnica - PUCRS}
\date{10 de março de 2023}

\begin{document}
\justifying

%-------------------------------------------------------
\begin{frame}
	\titlepage
\end{frame}

%=======================================================
\section{Autódromo: Desafios (Introdução)}

%-------------------------------------------------------
\begin{frame}\frametitle{Desafio 1}
Computadores são máquinas de propósito específico.\\
~\\
~\\
Programa é uma sequência de instruções que diz ao computador como realizar determinada tarefa.
\end{frame}

%-------------------------------------------------------
\begin{frame}\frametitle{Desafio 2}
\emph{Firmware} é o nome que se dá coletivamente ao computador e aos dispositivos periféricos.\\
~\\
~\\
\emph{Middleware} é o nome que se dá coletivamente ao conjunto de programas que um computador executa.
\end{frame}

%-------------------------------------------------------
\begin{frame}\frametitle{Desafio 3}
O processador ou UCP (Unidade Central de Processamento ou CPU) é capaz de realizar operações aritméticas, buscar dados na memória externa e salvar dados processados em unidades de armazenamento.\\
~\\
~\\
A memória primária armazena informações em circuitos eletrônicos de forma volátil (ou seja, apenas enquanto os circuitos estão energizados).
\end{frame}

%-------------------------------------------------------
\begin{frame}\frametitle{Desafio 4}
A memória secundária é formada por dispositivos de armazenamento que, em relação à memória principal, tem como característica o armazenamento de informação de forma volátil.\\
~\\
~\\
Compiladores traduzem um programa em linguagem de alto nível para instruções em linguagem de máquina que o computador é capaz de executar.
\end{frame}

%-------------------------------------------------------
\begin{frame}\frametitle{Desafio 5}
Interpretadores analisam e executam o programa em linguagem de alto nível instrução por instrução.\\
~\\
~\\
Interpretadores detectam erros de lógica antes do programa ser executado.
\end{frame}

%-------------------------------------------------------
\begin{frame}\frametitle{Desafio 6}
O compilador Java gera bytecodes que devem ser interpretados por uma Máquina Virtual Java.\\
~\\
~\\
O programa responsável pela compilação dos programas em Java chama-se \texttt{javac}.
\end{frame}

%-------------------------------------------------------
\begin{frame}\frametitle{Desafio 7}
O compilador Java converte os arquivos com a extensão \texttt{.java} para arquivos com a extensão \texttt{.bytecode}.\\
~\\
~\\
Nos programas em Java, os blocos fundamentais de construção de aplicações são chamados de classes.
\end{frame}

%-------------------------------------------------------
\begin{frame}\frametitle{Desafio 8}
Os erros detectados durante o processo de compilação são chamados de erros semânticos.\\
~\\
~\\
Os erros que ocorrem durante a execução e que geram mensagens da Máquina Virtual Java são chamados erros de lógica.
\end{frame}

%-------------------------------------------------------
\begin{frame}\frametitle{Desafio 9}
Um algoritmo é um conjunto finito e não ambíguo de passos que descreve detalhadamente como chegar à solução de um problema.\\
~\\
~\\
Em um fluxograma, processos a serem executados como atribuições são representados por retângulos.
\end{frame}

%-------------------------------------------------------
\begin{frame}\frametitle{Desafio 10}
Em um fluxograma, decisões são representadas por losangos.\\
~\\
~\\
Mapas Conceituais são um dos formalismos que poderiam ser usados para representar o algoritmo para a solução de um problema.
\end{frame}

%-------------------------------------------------------
\begin{frame}\frametitle{Desafio 11}
Em Java, dentro de uma classe pode-se encontrar métodos.\\
~\\
~\\
Métodos são trechos de código que desempenham determinada tarefa.	
\end{frame}

%-------------------------------------------------------
\begin{frame}\frametitle{Desafio 12}
Em Java, as sentenças (comandos) que especificam o que deve ser executado são colocadas dentro da classe e fora dos métodos.\\
~\\
~\\
Em Java, o método especial pelo qual a execução é iniciada chama-se \texttt{main}.
\end{frame}

%-------------------------------------------------------
\begin{frame}\frametitle{Desafio 13}
Em Java, comentários servem para documentar o código-fonte, tornando-o mais fácil de entender.\\
~\\
~\\
Em Java, textos ou \emph{strings} são delimitados por apóstrofos.
\end{frame}

%-------------------------------------------------------
\begin{frame}\frametitle{Desafio 14}
Em Java, comentários de uma linha iniciam com uma contra-barra seguida de outra contra-barra (\texttt{\textbackslash\textbackslash}).\\
~\\
~\\
Na programação em Java, cada comando deve ser colocado em uma linha separada.
\end{frame}

%-------------------------------------------------------
\begin{frame}\frametitle{Desafio 15}
Em Java, para imprimir aspas em um texto, deve-se substituir o caractere aspas que se deseja imprimir por dois apóstrofos.\\
~\\
~\\
Para imprimir $n$ contra-barras em um texto em Java, deve-se usar $n+1$ contra-barras.
\end{frame}

%=======================================================
\end{document}

