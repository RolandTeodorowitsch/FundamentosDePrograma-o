\documentclass[onecolumn,a4paper,10pt]{report}
%\documentclass[12pt,a4paper,twoside]{book} %twoside distingue página par de ímpar
\usepackage[utf8]{inputenc}
\usepackage[portuges]{babel} %para separar sílabas em Português, etc...
\usepackage[usenames,dvipsnames]{color} % para letras e caixas coloridas
\usepackage{latexsym} %para fazer $\Box$ no \LaTeX2$\epsilon$
\usepackage{makeidx} % índice remissivo
\usepackage{amstext} %texto em equações: $... \text{} ...$
\usepackage{float}
\usepackage{theorem}
\usepackage{tabularx} %tabelas ocupando toda a página
\usepackage[all]{xy}
\usepackage{a4wide} %correta formatação da página em A4
\usepackage{indentfirst} %adiciona espaços no primeiro parágrafo

\usepackage{graphics,amssymb,amsfonts,amsmath}
\usepackage{tikz}
\usepackage{enumerate,hyperref}
\usepackage{palatino}
\usepackage{ragged2e}
\usepackage{minted}
\usepackage{booktabs}
\usepackage{verbatim}
\usepackage[export]{adjustbox}
\usepackage{tikz}                   
\usepackage{xcolor}
\usepackage{textcomp} % para usar \textdegree
\usepackage{setspace}
\usetikzlibrary{shadows}

\newminted{java}{bgcolor=cyan!10}

\definecolor{cinza}{gray}{.8}
\definecolor{branco}{gray}{1}
\definecolor{preto}{gray}{0}
\definecolor{verdemusgo}{rgb}{.3,.7,.5}
\definecolor{vinho}{cmyk}{0,1,1,.5}
%\setcounter{secnumdepth}{1}
%\renewcommand{\thesection}{\textcolor{preto}{\arabic{section}}}
%\renewcommand{\thepage}{\textcolor{preto}{\color{preto}{{\scriptsize}}}}
{\theorembodyfont{\upshape}
\newtheorem{Dem}{Demonstração}[chapter]}
\newtheorem{Ex}{Exemplo}[chapter]
\newtheorem{Exer}{Exercício}
\newtheorem{Lista}{Lista de exercícios}
\newtheorem{Def}{Definição}[chapter]

\newtheorem{Pro}{Proposição}[chapter]
\newtheorem{Ax}{Axioma}[chapter]
\newtheorem{Teo}{Teorema}[chapter]
\newtheorem{Cor}{Corolário}[chapter]
\newtheorem{Cas}{Caso}[subsection]
\newtheorem{lema}{Lema}[chapter]
\newtheorem{que}{Questão}[chapter]
\newcommand{\dem}{\noindent{\bf Demonstração:}}
\newcommand{\sol}{\noindent{\it Solução.}}
\newcommand{\nota}{\noindent{\bf Notação:}}
\newcommand{\ex}{\noindent{\bf Exemplos}}
\newcommand{\Obs}{\noindent{\bf Observação:}}
\newcommand{\fim}{\hfill $\blacksquare$}
\newcommand{\ig}{\,\, = \,\,}
\newcommand{\+}{\, + \,}
\newcommand{\m}{\, - \,}
\newcommand{\I}{\mbox{$I\kern-0.40emI$}}
\newcommand{\Z}{\mbox{Z$\kern-0.40em$Z}}
\newcommand{\Q}{\mbox{I$\kern-0.60em$Q}}
\newcommand{\C}{\mbox{I$\kern-0.60em$C}}
\newcommand{\N}{\mbox{I$\kern-0.40em$N}}
\newcommand{\R}{\mbox{I$\kern-0.40em$R}}
\newcommand{\Ro}{\rm{I\!R\!}}
\newcommand{\disp}{\displaystyle}
\newcommand{\<}{\hspace*{-0.4cm}}
\newcommand{\ds}{\displaystyle}
\newcommand{\ov}{\overline}
\newcommand{\aj}{\vspace*{-0.2cm}}
\newcommand{\pt}{\hspace{-1mm}\times\hspace{-1mm}}
\newcommand{\cm}{\mbox{cm}}
\newcommand{\np}{\mbox{$\in \kern-0.80em/$}}
\newcommand{\tg} {\mbox{tg\,}}
\newcommand{\ptm}{\hspace{-0.4mm}\cdot\hspace{-0.4mm}}
\newcommand{\arc}{\stackrel{\;\;\frown}}
\newcommand{\rad}{\;\mbox{rad}}
\newcommand{\esp}{\;\;\;\;}
\newcommand{\sen}{\mbox{sen\,}}
\newcommand{\grau}{^{\mbox{{\scriptsize o}}}}
\newcommand{\real} {\mbox{$I\kern-0.60emR$}}
\newcommand{\vetor}{\stackrel{\color{vinho}\vector(1,0){15}}}
\newcommand{\arctg}{\mbox{arctg\,}}
\newcommand{\arcsen}{\mbox{arcsen\,}}
\newcommand{\ordinal}{^{\underline{\scriptsize\mbox{\rmo}}}}
\newcommand{\segundo}{$2^{\underline{o}}$ }
\newcommand{\primeiro}{$1^{\underline{o}}$ }
\newcommand{\nee}{\mbox{$\;=\kern-0.90em/\;$}}

\setlength{\parskip}{0.0cm} %espaco entre parágrafos
\setlength{\oddsidemargin}{-1cm} %margem esquerda das páginas
%\setlength{\unitlength}{3cm} %tamanho da figura criada
\linespread{1.5} %distância entre linhas
\setlength{\textheight}{25cm} %distância entre a primeira e última linha do texto(comprimento do texto)
\setlength{\textwidth}{18cm} %indica a largura do texto
\topmargin=-2cm %margem superior entre topo da página e o cabeçalho
%\headsep=0.5cm %distãncia entre o cabeçalho e o corpo do texto
%\setlength{\footskip}{27pt} %distãncia da última linha ao número da página
%\evensidemargin=-0.2in %margem esquerda das páginas pares
%\marginparwidth=1.7in %tamanho das notas de margem
%\marginparsep=0.2in %distância entre a margem direita e as notas de margem
%\topmargin=0cm
%\stackrel{\frown}{AB}

\begin{document}
\singlespacing

\begin{center}
Pontifícia Universidade Católica do Rio Grande do Sul (PUCRS)\\
Escola Politécnica\\
Disciplina: Fundamentos de Programação - Professor: Roland Teodorowitsch\\
12 de maio de 2023
\end{center}

\begin{center}
\textbf{Lista de Exercícios 2 - Unidade 5: Métodos}
\end{center}

\begin{enumerate}

	\item O número de Fibonacci de ordem \texttt{n} pode ser definido como a soma dos dois números de Fibonacci anteriores. Ou seja:\\
\texttt{Fibonacci(n) = Fibonacci(n-1) + Fibonacci(n-2)}\\
e, por definição, \texttt{Fibonacci(1) = 0} e \texttt{Fibonacci(2) = 1}.\\
Implemente, em Java:
\begin{itemize}
\item Um método NÃO recursivo que recebe \texttt{n}, calculando e retornando o número de Fibonacci de ordem \texttt{n};
\item Um método recursivo que recebe \texttt{n}, calculando e retornando o número de Fibonacci de ordem \texttt{n}.
\end{itemize}

	\item Considerando a implementação recursiva dos números de Fibonacci,  caso seja necessário usar esse método para calcular o número de Fibonacci de ordem 7, identifique:
\begin{itemize}
\item Quantas vezes o método fibonacci recursivo será chamado?
\item Quantas vezes o método fibonacci recursivo será chamado para \texttt{n} valendo 5?
\end{itemize}

	\item Implemente um método recursivo em Java que recebe \texttt{n} (inteiro e maior ou igual a 1) e que imprima os números inteiros de 1 até \texttt{n}, inclusive.

	\item Implemente um método recursivo em Java que recebe \texttt{n} (inteiro e maior ou igual a 1) e que imprima os números inteiros de \texttt{n} até 1, inclusive.

	\item Dado um valor inteiro e positivo (\texttt{n}), o valor da constante de \emph{Euler} poder ser calculando com precisão diretamente proporcional a \texttt{n} através da fórmula:
\[ E = \frac{1}{0!} + \frac{1}{1!} + \frac{1}{2!}+ \frac{1}{3!} + ... + \frac{1}{n!}\]
Implemente, em Java:
\begin{itemize}
\item Um método NÃO recursivo que recebe \texttt{n}, retornando o valor de \emph{Euler} calculado usando a fórmula acima;
\item Um método recursivo que recebe \texttt{n}, retornando o valor de \emph{Euler} calculado usando a fórmula acima.
\end{itemize}

	\item Considere o programa em Java a seguir e mostre o que será impresso, respeitando a ordem de execução.
{\scriptsize
\begin{javacode}
public class Teste2 {
    public static int a = 2;
    public static int b = 10;
    
    public static void metodo1(int b) {
        System.out.println("[1] "+a);
        int a = b;
        System.out.println("[2] "+a);
        if (b==1)
            return;
        else
            metodo1(b-1);
    }

    public static void metodo2() {
        System.out.println("[3] "+a);
        System.out.println("[4] "+b);
        metodo1(a);
        ++b;
        int b = 2;
        System.out.println("[5] "+a);
        System.out.println("[6] "+b);        
    }

    public static void main(String[] args) {
        System.out.println("[7] "+a);
        a++;
        int a = 15;
        System.out.println("[8] "+a);
        a++;
        metodo2();
        System.out.println("[9] "+a);        
    }    
}
\end{javacode}
}
\end{enumerate}
\end{document}
