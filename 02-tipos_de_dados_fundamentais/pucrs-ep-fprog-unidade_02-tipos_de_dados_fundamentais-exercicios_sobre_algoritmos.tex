\documentclass[onecolumn,a4paper,10pt]{report}
%\documentclass[12pt,a4paper,twoside]{book} %twoside distingue página par de ímpar
\usepackage[utf8]{inputenc}
\usepackage[portuges]{babel} %para separar sílabas em Português, etc...
\usepackage[usenames,dvipsnames]{color} % para letras e caixas coloridas
\usepackage{latexsym} %para fazer $\Box$ no \LaTeX2$\epsilon$
\usepackage{makeidx} % índice remissivo
\usepackage{amstext} %texto em equações: $... \text{} ...$
\usepackage{theorem}
\usepackage{tabularx} %tabelas ocupando toda a página
\usepackage[all]{xy}
\usepackage{a4wide} %correta formatação da página em A4
\usepackage{indentfirst} %adiciona espaços no primeiro parágrafo

\usepackage{graphics,amssymb,amsfonts,amsmath}
\usepackage{tikz}
\usepackage{enumerate,hyperref}
\usepackage{palatino}
\usepackage{ragged2e}
\usepackage{minted}
\usepackage{booktabs}
\usepackage{verbatim}
\usepackage[export]{adjustbox}
\usepackage{tikz}                   
\usepackage{xcolor}
\usepackage{textcomp} % para usar \textdegree
\usepackage{setspace}
\usetikzlibrary{shadows}

\newminted{java}{bgcolor=cyan!10}

\definecolor{cinza}{gray}{.8}
\definecolor{branco}{gray}{1}
\definecolor{preto}{gray}{0}
\definecolor{verdemusgo}{rgb}{.3,.7,.5}
\definecolor{vinho}{cmyk}{0,1,1,.5}
%\setcounter{secnumdepth}{1}
%\renewcommand{\thesection}{\textcolor{preto}{\arabic{section}}}
%\renewcommand{\thepage}{\textcolor{preto}{\color{preto}{{\scriptsize}}}}
{\theorembodyfont{\upshape}
\newtheorem{Dem}{Demonstração}[chapter]}
\newtheorem{Ex}{Exemplo}[chapter]
\newtheorem{Exer}{Exercício}
\newtheorem{Lista}{Lista de exercícios}
\newtheorem{Def}{Definição}[chapter]

\newtheorem{Pro}{Proposição}[chapter]
\newtheorem{Ax}{Axioma}[chapter]
\newtheorem{Teo}{Teorema}[chapter]
\newtheorem{Cor}{Corolário}[chapter]
\newtheorem{Cas}{Caso}[subsection]
\newtheorem{lema}{Lema}[chapter]
\newtheorem{que}{Questão}[chapter]
\newcommand{\dem}{\noindent{\bf Demonstração:}}
\newcommand{\sol}{\noindent{\it Solução.}}
\newcommand{\nota}{\noindent{\bf Notação:}}
\newcommand{\ex}{\noindent{\bf Exemplos}}
\newcommand{\Obs}{\noindent{\bf Observação:}}
\newcommand{\fim}{\hfill $\blacksquare$}
\newcommand{\ig}{\,\, = \,\,}
\newcommand{\+}{\, + \,}
\newcommand{\m}{\, - \,}
\newcommand{\I}{\mbox{$I\kern-0.40emI$}}
\newcommand{\Z}{\mbox{Z$\kern-0.40em$Z}}
\newcommand{\Q}{\mbox{I$\kern-0.60em$Q}}
\newcommand{\C}{\mbox{I$\kern-0.60em$C}}
\newcommand{\N}{\mbox{I$\kern-0.40em$N}}
\newcommand{\R}{\mbox{I$\kern-0.40em$R}}
\newcommand{\Ro}{\rm{I\!R\!}}
\newcommand{\disp}{\displaystyle}
\newcommand{\<}{\hspace*{-0.4cm}}
\newcommand{\ds}{\displaystyle}
\newcommand{\ov}{\overline}
\newcommand{\aj}{\vspace*{-0.2cm}}
\newcommand{\pt}{\hspace{-1mm}\times\hspace{-1mm}}
\newcommand{\cm}{\mbox{cm}}
\newcommand{\np}{\mbox{$\in \kern-0.80em/$}}
\newcommand{\tg} {\mbox{tg\,}}
\newcommand{\ptm}{\hspace{-0.4mm}\cdot\hspace{-0.4mm}}
\newcommand{\arc}{\stackrel{\;\;\frown}}
\newcommand{\rad}{\;\mbox{rad}}
\newcommand{\esp}{\;\;\;\;}
\newcommand{\sen}{\mbox{sen\,}}
\newcommand{\grau}{^{\mbox{{\scriptsize o}}}}
\newcommand{\real} {\mbox{$I\kern-0.60emR$}}
\newcommand{\vetor}{\stackrel{\color{vinho}\vector(1,0){15}}}
\newcommand{\arctg}{\mbox{arctg\,}}
\newcommand{\arcsen}{\mbox{arcsen\,}}
\newcommand{\ordinal}{^{\underline{\scriptsize\mbox{\rmo}}}}
\newcommand{\segundo}{$2^{\underline{o}}$ }
\newcommand{\primeiro}{$1^{\underline{o}}$ }
\newcommand{\nee}{\mbox{$\;=\kern-0.90em/\;$}}

\setlength{\parskip}{0.0cm} %espaco entre parágrafos
\setlength{\oddsidemargin}{-1cm} %margem esquerda das páginas
%\setlength{\unitlength}{3cm} %tamanho da figura criada
\linespread{1.5} %distância entre linhas
\setlength{\textheight}{25cm} %distância entre a primeira e última linha do texto(comprimento do texto)
\setlength{\textwidth}{18cm} %indica a largura do texto
\topmargin=-2cm %margem superior entre topo da página e o cabeçalho
%\headsep=0.5cm %distãncia entre o cabeçalho e o corpo do texto
%\setlength{\footskip}{27pt} %distãncia da última linha ao número da página
%\evensidemargin=-0.2in %margem esquerda das páginas pares
%\marginparwidth=1.7in %tamanho das notas de margem
%\marginparsep=0.2in %distância entre a margem direita e as notas de margem
%\topmargin=0cm
%\stackrel{\frown}{AB}

\begin{document}
\singlespacing

\begin{center}
Pontifícia Universidade Católica do Rio Grande do Sul (PUCRS)\\
Curso de Bacharelado em Ciência da Computação\\
Disciplina: Fundamentos de Programação - Professor: Roland Teodorowitsch\\
5 de abril de 2020
\end{center}
~\\

\begin{center}
\textbf{Lista de Exercícios - Algoritmos Sequenciais (ORTH, 2001, p. 16-17)}
\end{center}
~\\

\begin{enumerate}[1.]

%1----------------------------------------------------------------------
\item Reescrever o algoritmo que calcula as médias, apresentado no início deste capítulo (ORTH, 2001, p. 11), utilizando a linguagem apresentada.\\
\fbox{
\begin{minipage}{0.75\textwidth} % 3/4 da largura de texto
\begin{enumerate}[1)]
	\item Obter os valores de a, b, c
	\item Calcular a média aritmética pela fórmula $ma = \frac{a+b+c}{3}$
	\item Calcular a média harmônica pela fórmula $mh = \frac{3}{\frac{1}{a}+\frac{1}{b}+\frac{1}{c}}$
	\item Calcular a média geométrica pela fórmula $mg = \sqrt[3]{a \times b \times c}$
	\item Calcular a média ponderada pela fórmula $mp = \frac{1 \times a + 2 \times b + 3 \times c}{1+2+3}$
	\item Comunicar os resultados obtidos: $ma$, $mh$, $mg$, $mp$
	\item Terminar
\end{enumerate}
\end{minipage}
} % fbox

%2----------------------------------------------------------------------
\item Escrever um algoritmo que lê o número de um funcionário, seu número de horas trabalhadas, o valor que recebe por hora, o número de filhos com idade menor do que 14 anos e o valor do salário família (pago por filho com menos de 14 anos). Calcular o salário total deste funcionário e escrever o seu número e o seu salário total.

%3----------------------------------------------------------------------
\item Escrever um algoritmo que calcula e escreve o fatorial de 5.

%4----------------------------------------------------------------------
\item Escrever um algoritmo que lê 3 valores reais \textbf{a}, \textbf{b}, \textbf{c} e calcula:
\begin{itemize}
    \item A área do triângulo que tem \textbf{a} por base e \textbf{b} por altura;
    \item A área do círculo de raio \textbf{c};
    \item A área do trapézio que tem \textbf{a} e \textbf{b} por bases e \textbf{c} por altura;
    \item A área do quadrado de lado \textbf{b};
    \item A área do retângulo de lados \textbf{a} e \textbf{b};
    \item A área da superfície de um cubo que tem \textbf{c} por aresta.
\end{itemize}

%5----------------------------------------------------------------------
\item Escrever um algoritmo que escreve os números ímpares entre 10 e 20.

%6----------------------------------------------------------------------
\item Escrever um algoritmo que lê \textbf{p}, \textbf{u} e \textbf{r} respectivamente o primeiro termo de uma progressão aritmética, o último termo da progressão e a sua razão. Determinar e escrever a soma dos termos desta progressão.

%7----------------------------------------------------------------------
\item Escrever um algoritmo que lê o número de peças do tipo 1, o valor de cada peça do tipo 1, o número de peças do tipo 2, o valor de cada peça do tipo 2 e o percentual do IPI a ser acrescentado. Calcular e escrever o valor total a ser pago por esta compra.

%8----------------------------------------------------------------------
\item Um avião em linha reta, a uma altitude \textbf{a} passa sobre um ponto \textbf{p} situado no solo, num instante \textbf{t} = 0. Se a velocidade é \textbf{v}, calcular a distância \textbf{d} do avião ao ponto \textbf{p} após 30 segundos. Escrever um algoritmo que lê \textbf{v} (em m/s) e \textbf{a} (em m) e calcula e escreve a distância do avião ao ponto \textbf{p} após 30 segundos.

%9----------------------------------------------------------------------
\item Uma farmácia paga o seu funcionário a cada sexta-feira e deseja deixar pronto o envelope de pagamento. Escrever um algoritmo que lê o valor do salário do funcionário e calcula qual o menor número possível de notas de 100, 50, 10, 5 e 1, em que o valor lido pode ser decomposto. Escrever o valor lido e o número de notas de cada tipo que compõe o envelope de pagamento.

%10----------------------------------------------------------------------
\item Escrever um algoritmo que lê o número de um vendedor, o seu salário fixo, o total de vendas por ele efetuadas e o percentual que ganha sobre o total de suas vendas. Calcular o salário total do vendedor e escrever o número e o salário do vendedor.

%11----------------------------------------------------------------------
\item Escrever um algoritmo que lê 3 valores \textbf{a}, \textbf{b}, \textbf{c} que são lados de um triângulo e calcula e escreve a área deste triângulo.\\
\emph{Área} $ = \sqrt{s \times (s - a) \times (s - b) \times (s - c)}$\\
Onde $s = $ \emph{semiperímetro}.

%12----------------------------------------------------------------------
\item O custo ao consumidor, de um carro novo, é a soma do custo de fábrica com a percentagem do distribuidor e o
percentual dos impostos (aplicados ao custo de fábrica). Escrever um algoritmo que lê o custo de fábrica, o
percentual do distribuidor e o percentual dos impostos e calcula e escreve o valor a ser pago pelo consumidor
por este carro.

%13----------------------------------------------------------------------
\item Uma revendedora de carros usados paga a seus funcionários vendedores, um salário fixo por mês, mais uma
comissão também fixa para cada carro vendido e mais 5\% do valor das vendas por ele efetuadas. Escrever um
algoritmo que lê o número do vendedor, o número de carros por ele vendidos, o valor de cada carro, o salário
fixo e o valor que recebe por carro vendido e calcula o salário a ser pago a este vendedor, escrevendo o número
do vendedor e seu salário total.

%14----------------------------------------------------------------------
\item Considerando que o aumento dos funcionários de uma empresa tenha sido definido da seguinte forma: 80\% de
um índice chamado INTEMP e mais um percentual de produtividade discutido com a empresa por ocasião do
dissídio da categoria. Escrever um algoritmo que lê o número do funcionário, seu salário atual, o valor do
INTEMP e o índice de produtividade conquistado e escreve o número do funcionário, seu aumento e seu novo
salário.

%15----------------------------------------------------------------------
\item Escrever um algoritmo que lê as coordenadas de dois pontos no plano cartesiano e calcula e escreve a distância
entre estes dois pontos, sabendo-se que a fórmula da distância entre dois pontos $P_1(X_1,Y_1)$ e $P_2(X_2,Y_2)$ é\\
\emph{Distância} $= \sqrt{ ( X_2 - X_1 )^2 + ( Y_2 - Y_1 )^2 }$.

%16----------------------------------------------------------------------
\item Escrever um algoritmo que lê 3 valores a, b, c e os escreve. Encontre, a seguir, o maior dos 3 valores e o
escreva com a mensagem: ``É O MAIOR''.
\[Maior = \frac{a + b + | a - b |}{2}\]

%17----------------------------------------------------------------------
\item Escrever um algoritmo que lê o valor de uma compra em dólares e a taxa do dólar no dia da compra e calcula e
escreve o valor a ser pago em reais.

%18----------------------------------------------------------------------
\item Um hotel com 42 apartamentos resolveu fazer promoções para os fins de semana for a da alta temporada, isto
é, nos meses de abril, maio, junho, agosto, setembro outubro e novembro. A taxa da promoção é de 22\% da
diária normal. A expectativa é aumentar a taxa de ocupação de 40 para 70\%. Supondo que as expectativas se
confirmem, escrever um algoritmo que lê a diária normal e calcula e escreve as seguintes informações:
\begin{enumerate}[a)]
	\item O valor da diária no período da promoção.
	\item O valor médio arrecadado sem a promoção, durante um mês.
	\item O valor médio arrecadado com a promoção, durante um mês.
	\item O lucro ou prejuízo mensal com a promoção.
\end{enumerate}

%19----------------------------------------------------------------------
\item Escrever um algoritmo que lê a hora de início de um jogo e a hora de fim do jogo, considerando apenas horas
inteiras e jogos que começam e terminam no mesmo dia, calcular e escrever o tempo de duração do jogo em
horas.

%20----------------------------------------------------------------------
\item O mesmo problema anterior, mas escrevendo o tempo do jogo em minutos.

%21----------------------------------------------------------------------
\item Escrever um algoritmo que lê um número de 3 dígitos e o inverte, escrevendo o número lido e o número invertido.

\end{enumerate}

%----------------------------------------------------------------------
~\\
\noindent{\textbf{REFERÊNCIAS}}

\noindent{ORTH, Afonso Inácio. \textbf{Algoritmos e Programação com Resumo das Linguagens PASCAL e C}. Porto Alegre: AIO, 2001. 176 p.}

\end{document}

