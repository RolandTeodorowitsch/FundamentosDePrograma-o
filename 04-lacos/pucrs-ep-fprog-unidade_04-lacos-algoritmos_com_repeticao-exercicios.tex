\documentclass[onecolumn,a4paper,10pt]{report}
%\documentclass[12pt,a4paper,twoside]{book} %twoside distingue página par de ímpar
\usepackage[utf8]{inputenc}
\usepackage[portuges]{babel} %para separar sílabas em Português, etc...
\usepackage[usenames,dvipsnames]{color} % para letras e caixas coloridas
\usepackage{latexsym} %para fazer $\Box$ no \LaTeX2$\epsilon$
\usepackage{makeidx} % índice remissivo
\usepackage{amstext} %texto em equações: $... \text{} ...$
\usepackage{theorem}
\usepackage{tabularx} %tabelas ocupando toda a página
\usepackage[all]{xy}
\usepackage{a4wide} %correta formatação da página em A4
\usepackage{indentfirst} %adiciona espaços no primeiro parágrafo

\usepackage{graphics,amssymb,amsfonts,amsmath}
\usepackage{tikz}
\usepackage{enumerate,hyperref}
\usepackage{palatino}
\usepackage{ragged2e}
\usepackage{minted}
\usepackage{booktabs}
\usepackage{verbatim}
\usepackage[export]{adjustbox}
\usepackage{tikz}                   
\usepackage{xcolor}
\usepackage{textcomp} % para usar \textdegree
\usepackage{setspace}
\usetikzlibrary{shadows}

\newminted{java}{bgcolor=cyan!10}

\definecolor{cinza}{gray}{.8}
\definecolor{branco}{gray}{1}
\definecolor{preto}{gray}{0}
\definecolor{verdemusgo}{rgb}{.3,.7,.5}
\definecolor{vinho}{cmyk}{0,1,1,.5}
%\setcounter{secnumdepth}{1}
%\renewcommand{\thesection}{\textcolor{preto}{\arabic{section}}}
%\renewcommand{\thepage}{\textcolor{preto}{\color{preto}{{\scriptsize}}}}
{\theorembodyfont{\upshape}
\newtheorem{Dem}{Demonstração}[chapter]}
\newtheorem{Ex}{Exemplo}[chapter]
\newtheorem{Exer}{Exercício}
\newtheorem{Lista}{Lista de exercícios}
\newtheorem{Def}{Definição}[chapter]

\newtheorem{Pro}{Proposição}[chapter]
\newtheorem{Ax}{Axioma}[chapter]
\newtheorem{Teo}{Teorema}[chapter]
\newtheorem{Cor}{Corolário}[chapter]
\newtheorem{Cas}{Caso}[subsection]
\newtheorem{lema}{Lema}[chapter]
\newtheorem{que}{Questão}[chapter]
\newcommand{\dem}{\noindent{\bf Demonstração:}}
\newcommand{\sol}{\noindent{\it Solução.}}
\newcommand{\nota}{\noindent{\bf Notação:}}
\newcommand{\ex}{\noindent{\bf Exemplos}}
\newcommand{\Obs}{\noindent{\bf Observação:}}
\newcommand{\fim}{\hfill $\blacksquare$}
\newcommand{\ig}{\,\, = \,\,}
\newcommand{\+}{\, + \,}
\newcommand{\m}{\, - \,}
\newcommand{\I}{\mbox{$I\kern-0.40emI$}}
\newcommand{\Z}{\mbox{Z$\kern-0.40em$Z}}
\newcommand{\Q}{\mbox{I$\kern-0.60em$Q}}
\newcommand{\C}{\mbox{I$\kern-0.60em$C}}
\newcommand{\N}{\mbox{I$\kern-0.40em$N}}
\newcommand{\R}{\mbox{I$\kern-0.40em$R}}
\newcommand{\Ro}{\rm{I\!R\!}}
\newcommand{\disp}{\displaystyle}
\newcommand{\<}{\hspace*{-0.4cm}}
\newcommand{\ds}{\displaystyle}
\newcommand{\ov}{\overline}
\newcommand{\aj}{\vspace*{-0.2cm}}
\newcommand{\pt}{\hspace{-1mm}\times\hspace{-1mm}}
\newcommand{\cm}{\mbox{cm}}
\newcommand{\np}{\mbox{$\in \kern-0.80em/$}}
\newcommand{\tg} {\mbox{tg\,}}
\newcommand{\ptm}{\hspace{-0.4mm}\cdot\hspace{-0.4mm}}
\newcommand{\arc}{\stackrel{\;\;\frown}}
\newcommand{\rad}{\;\mbox{rad}}
\newcommand{\esp}{\;\;\;\;}
\newcommand{\sen}{\mbox{sen\,}}
\newcommand{\grau}{^{\mbox{{\scriptsize o}}}}
\newcommand{\real} {\mbox{$I\kern-0.60emR$}}
\newcommand{\vetor}{\stackrel{\color{vinho}\vector(1,0){15}}}
\newcommand{\arctg}{\mbox{arctg\,}}
\newcommand{\arcsen}{\mbox{arcsen\,}}
\newcommand{\ordinal}{^{\underline{\scriptsize\mbox{\rmo}}}}
\newcommand{\segundo}{$2^{\underline{o}}$ }
\newcommand{\primeiro}{$1^{\underline{o}}$ }
\newcommand{\nee}{\mbox{$\;=\kern-0.90em/\;$}}

\setlength{\parskip}{0.0cm} %espaco entre parágrafos
\setlength{\oddsidemargin}{-1cm} %margem esquerda das páginas
%\setlength{\unitlength}{3cm} %tamanho da figura criada
\linespread{1.5} %distância entre linhas
\setlength{\textheight}{25cm} %distância entre a primeira e última linha do texto(comprimento do texto)
\setlength{\textwidth}{18cm} %indica a largura do texto
\topmargin=-2cm %margem superior entre topo da página e o cabeçalho
%\headsep=0.5cm %distãncia entre o cabeçalho e o corpo do texto
%\setlength{\footskip}{27pt} %distãncia da última linha ao número da página
%\evensidemargin=-0.2in %margem esquerda das páginas pares
%\marginparwidth=1.7in %tamanho das notas de margem
%\marginparsep=0.2in %distância entre a margem direita e as notas de margem
%\topmargin=0cm
%\stackrel{\frown}{AB}

\begin{document}
\singlespacing

\begin{center}
Pontifícia Universidade Católica do Rio Grande do Sul (PUCRS)\\
Escola  Politécnica\\
Disciplina: Fundamentos de Programação - Professor: Roland Teodorowitsch\\
24 de agosto de 2022
\end{center}
~\\

\begin{center}
\textbf{Lista de Exercícios - Algoritmos Com Repetição (ORTH, 2001, p. 34-39)}
\end{center}
~\\

\begin{enumerate}[1.]

%1----------------------------------------------------------------------
\item Escrever um algoritmo que lê 5 valores reais para \texttt{a}, um de cada vez, e conta quantos destes valores são negativos, escrevendo esta informação.\\
{\tiny Adaptado de: Orth (2001, p. 34)}

%2----------------------------------------------------------------------
\item Escrever um algoritmo que gera e escreve os números ímpares entre 100 e 200.\\
{\tiny Adaptado de: Orth (2001, p. 34)}

%3----------------------------------------------------------------------
\item Escrever um algoritmo que lê 10 valores reais, um de cada vez, e conta quantos deles estão no intervalo [10,20] e quantos deles estão fora deste intervalo, escrevendo estas informações.\\
{\tiny Adaptado de: Orth (2001, p. 34)}

%4----------------------------------------------------------------------
\item Escrever um algoritmo que lê um número não conhecido de valores reais, um de cada vez, e conta quantos deles estão em cada um dos intervalos [0,25), [25,50), [50,75) e [75,100], escrevendo estas informações. A leitura deve ser feita até que um valor fora dos intervalos seja fornecido.\\
{\tiny Adaptado de: Orth (2001, p. 34)}

%5----------------------------------------------------------------------
\item Escrever um algoritmo semelhante ao anterior, que calcula as médias aritméticas de cada intervalo e as escreve, juntamente com o número de valores encontrados em cada intervalo. Caso nenhum valor tenha sido encontrado em determinado intervalo, escrever ``*'' no lugar da média.\\
{\tiny Adaptado de: Orth (2001, p. 34)}

%6----------------------------------------------------------------------
\item A série de Fibonacci tem como dados os 2 primeiros termos da série que são respectivamente 0 e 1. À partir deles, os demais termos são construídos pela seguinte regra:
\[ t_{n} = t_{n-1} + t_{n-2} \]
Escrever um algoritmo que gera os 10 primeiros termos desta série e calcula e escreve a sua soma.\\
{\tiny Adaptado de: Orth (2001, p. 34)}

%7----------------------------------------------------------------------
\item Escrever um algoritmo que gera os 10 primeiros termos da série de Fibonacci, escrevendo para cada termo gerado o número de ordem e o valor do termo da série de Fibonacci. Considere que os números de ordem iniciam com 0.\\
{\tiny Adaptado de: Orth (2001, p. 34)}

%8----------------------------------------------------------------------
\item Escrever um algoritmo que gera os 30 primeiros termos da série de Fibonacci e escreve os termos gerados com a mensagem: ``EH PRIMO'' ou ``NAO EH PRIMO'', conforme o caso.\\
{\tiny Adaptado de: Orth (2001, p. 34)}

%9----------------------------------------------------------------------
\item Escrever um algoritmo que lê um conjunto não determinado de valores, um de cada vez, e escreve uma tabela, contendo 20 linhas em cada página. A tabela deve conter o valor lido, seu quadrado, seu cubo e sua raiz quadrada.\\
{\tiny Adaptado de: Orth (2001, p. 34)}

%10----------------------------------------------------------------------
\item Escrever um algoritmo que lê um número não determinado de valores para \texttt{m}, todos inteiros e positivos, um de cada vez. Se \texttt{m} for par, verificar quantos divisores possui e escrever esta informação. Se \texttt{m} for ímpar e menor do que 12 calcular e escrever o fatorial de \texttt{m}. Se \texttt{m} for ímpar e maior do que 12, calcular e escrever a soma dos inteiros de 1 até \texttt{m}.\\
{\tiny Adaptado de: Orth (2001, p. 34)}

%11----------------------------------------------------------------------
\item Escrever um algoritmo que lê um número não determinado de valores \texttt{a}, todos inteiros e positivos, um de cada vez (até que um valor negativo seja lido), e calcule e escreva a média aritmética dos valores lidos, bem como, a quantidade de valores pares, a quantidade de valores ímpares, a percentagem dos valores pares e a percentagem dos valores ímpares. Se nenhum valor positivo for fornecido, imprimir ``*'' no lugar da média e das percentagens.\\
{\tiny Adaptado de: Orth (2001, p. 34)}

%12----------------------------------------------------------------------
\item Escrever um algoritmo que escreve os números primos entre 100 e 200, bem como, a soma destes números primos.\\
{\tiny Adaptado de: Orth (2001, p. 34)}

%13----------------------------------------------------------------------
\item Escrever um algoritmo que lê 10 valores para \texttt{n}, um de cada vez, todos inteiros e positivos, e para cada \texttt{n} lido, escreva a tabuada de 1 até \texttt{n} de \texttt{n}.\\
\fbox{
\begin{minipage}{0.12\textwidth} % 3/4 da largura de texto
1 x n = n\\
2 x n = 2n\\
. . . . .\\
. . . . .\\
. . . . .\\
n x n = n2
\end{minipage}
} % fbox
\\
{\tiny Adaptado de: Orth (2001, p. 35)}

%14----------------------------------------------------------------------
\item Escrever um algoritmo que lê 5 pares de valores (\texttt{a}, \texttt{b}), todos inteiros e positivos, um par de dada vez, com $a < b$ e escreve os inteiros pares de \texttt{a} até \texttt{b}, incluindo \texttt{a} e \texttt{b}, se eles forem pares.\\
{\tiny Adaptado de: Orth (2001, p. 35)}

%15----------------------------------------------------------------------
\item Escrever um algoritmo que lê um número não determinado de pares de valores (\texttt{m}, \texttt{n}), todos inteiros e positivos, um par de cada vez, e calcula e escreve a soma dos \texttt{n} inteiros consecutivos à partir de \texttt{m} inclusive.\\
{\tiny Adaptado de: Orth (2001, p. 35)}

%16----------------------------------------------------------------------
\item Escrever um algoritmo que lê um número não determinado de conjuntos de valores, cada um formado pelo número do aluno (um valor inteiro) e suas 3 notas. O algoritmo deve encerrar quando um número negativo for fornecido como número de aluno. Calcular, para cada aluno, a média ponderada com pesos respectivos de 4 para a maior nota e peso 3 para as outras duas. Escrever o número do aluno, a média calculada e uma mensagem ``APROVADO'' se a nota for $\geq 5$ ou ``REPROVADO'' se nota $< 5$.\\
{\tiny Adaptado de: Orth (2001, p. 35)}

%17----------------------------------------------------------------------
\item Escrever um algoritmo que lê um valor \texttt{n}, inteiro e positivo, e que calcula e escreve o valor de \texttt{E}, onde:
\[E = 1 + \frac{1}{1!} + \frac{1}{2!} + \frac{1}{3!} + ... + \frac{1}{n!} \]\\
{\tiny Adaptado de: Orth (2001, p. 35)}

%18----------------------------------------------------------------------
\item Um Clube de Futebol de uma cidade fez uma pesquisa entre seus sócios, coletando dados sobre o salário e o número de filhos dos sócios. O Clube deseja saber:
\begin{enumerate}[a)]
    \item A média do salário dos sócios;
    \item A média do número de filhos;
    \item O maior salário;
    \item O percentual de pessoas com salário até R\$400,00.
\end{enumerate}
O final da leitura de dados se dará quando da leitura de um salário negativo. Caso nenhuma informação seja fornecida, imprimir ``* * * * ''.\\
{\tiny Adaptado de: Orth (2001, p. 35)}

%19----------------------------------------------------------------------
\item Gustavo tem \texttt{a1} metros e cresce \texttt{c1} centímetros por ano, enquanto Juliano tem \texttt{a2} metros e cresce \texttt{c2} centímetros por ano. Construa um algoritmo que calcula e escreve, quantos anos serão necessários para que Juliano ultrapasse Gustavo. Caso não seja possível que Juliano ultrapasse a altura de Gustavo, escreva ``*''. Use laços para resolver o problema.\\
{\tiny Adaptado de: Orth (2001, p. 35)}

%20----------------------------------------------------------------------
\item Em uma eleição presidencial existem 4 candidatos. Os votos são informados por meio de códigos. Os dados utilizados para a contagem dos votos obedecem à seguinte codificação:
\begin{itemize}
    \item Os códigos 1, 2, 3 e 4 representam os votos para os respectivos candidatos.
    \item O código 5 representa o voto nulo.
    \item O código 6 representa o voto em branco.
\end{itemize}
Elabore um algoritmo que lê os votos, um de cada vez, e calcula e escreve:
\begin{itemize}
    \item o total de votos para cada candidato;
    \item  o total de votos nulos;
    \item o percentual de votos em branco.
\end{itemize}
{\tiny Adaptado de: Orth (2001, p. 35-36)}

%21----------------------------------------------------------------------
\item Escrever um algoritmo que lê um conjunto de 15 valores, um de cada vez, acompanhados de um código 1 ou 2. O valor representa o número de cobaias utilizadas em cada uma das 15 experiências feitas, e os códigos 1 ou 2 identificam se a experiência era, respectivamente, com coelhos ou ratos. Quer-se saber o número total de cobaias utilizadas, o total de coelhos, o total de ratos, a percentagem de coelhos e a percentagem de ratos. Escrever estes valores.\\
{\tiny Adaptado de: Orth (2001, p. 36)}

%22----------------------------------------------------------------------
\item Escrever um algoritmo que lê, para cada um dos vendedores de uma empresa, o seu número de identificação (inteiro), seu salário fixo e o total das vendas por ele efetuadas em reais. Cada vendedor recebe um salário fixo e uma comissão proporcional às vendas por ele efetuadas. A comissão é de 3\% sobre o total de vendas até 10000,00 e 5\% sobre o que ultrapassa este valor. Escrever, para cada vendedor, o seu número de identificação e o seu salário total. O processamento deve ser feito até que um número negativo de funcionário seja fornecido.\\
{\tiny Adaptado de: Orth (2001, p. 36)}

%23----------------------------------------------------------------------
\item Escrever um algoritmo que lê um número \texttt{n} (que representa o número de termos de uma Progressão Aritmética), \texttt{a1} (o primeiro termo desta progressão) e \texttt{r} (a razão desta progressão). Escrever os \texttt{n} termos desta progressão, bem como a sua soma.\\
{\tiny Adaptado de: Orth (2001, p. 36)}

%24----------------------------------------------------------------------
\item Escrever um algoritmo que lê 5 conjuntos de 2 valores, o primeiro representando o número de um aluno (valor inteiro), e o segundo representando a sua altura em centímetros (valor real). Encontrar o aluno mais alto e o mais baixo, escrevendo: o número do aluno mais alto, a mensagem `` eh o mais alto com '' e a altura mais alta; e o número do aluno mais baixo, a mensagem `` eh o mais baixo com '' e a altura mais baixa. Em caso de empate, considerar sempre o número do aluno que foi lido antes.\\
{\tiny Adaptado de: Orth (2001, p. 36)}

%25----------------------------------------------------------------------
\item Um avião voando em linha reta, a uma altitude \texttt{a}, passa sobre um ponto \texttt{P} situado em solo num instante \texttt{t = 0}. Se a velocidade \texttt{v} e a altura \texttt{a} forem lidos, calcular a distância \texttt{d} do avião ao ponto \texttt{P} após 1, 2, ..., 30
segundos. Escrever, para cada \texttt{t}, o tempo e a distância.\\
{\tiny Adaptado de: Orth (2001, p. 36)}

%26----------------------------------------------------------------------
\item Escrever um algoritmo que lê um valor \texttt{n} que indica quantos valores devem ser lidos para \texttt{m}, valores todos inteiros e positivos, com leitura de um valor de cada vez. Escreva, para cada valor lido, o próprio valor lido, seu fatorial e sua raiz cúbica.\\
{\tiny Adaptado de: Orth (2001, p. 36)}

%27----------------------------------------------------------------------
\item Escrever um algoritmo que gera e escreve os 5 primeiros números perfeitos. Um número perfeito é aquele que é igual a soma de seus divisores, exceto ele próprio. Exemplos:\\
$6 = 1 + 2 + 3$\\
$28 = 1 + 2 + 4 + 7 + 14$\\
{\tiny Adaptado de: Orth (2001, p. 36)}

%28----------------------------------------------------------------------
\item Escrever um algoritmo que lê 50 valores, um de cada vez, e encontra e escreve o maior deles.\\
{\tiny Adaptado de: Orth (2001, p. 36)}

%29----------------------------------------------------------------------
\item Escrever um algoritmo lê \texttt{a}, \texttt{b}, \texttt{c}, \texttt{d} e \texttt{e}, que constituem o Gabarito de uma prova de 5 questões. Leia, a seguir, um número não determinado de conjuntos de 6 valores: \texttt{num}, \texttt{a1}, \texttt{a2}, \texttt{a3}, \texttt{a4} e \texttt{a5}, onde \texttt{num} representa o número do aluno e os demais valores são as respostas daquele aluno às 5 questões da prova. Conte o número de acertos e multiplique por 2. Escrever, para cada aluno, o seu número e a sua nota.\\
{\tiny Adaptado de: Orth (2001, p. 36)}

%30----------------------------------------------------------------------
\item Foi feita uma pesquisa em um município. A pesquisa coletou, para cada habitante, os seguintes dados: idade, sexo (``M'' ou ``F'') e salário. Escreva um algoritmo que lê os conjuntos de dados obtidos, um conjunto de cada vez, obtém e escreve as seguintes informações:
\begin{enumerate}[a)]
    \item A média de salário dos habitantes do município.
    \item A maior e a menor idade das pessoas pesquisadas.
    \item A quantidade de mulheres com salário até 500,00.
\end{enumerate}
O algoritmo deve terminar quando for fornecida uma idade negativa.\\
{\tiny Adaptado de: Orth (2001, p. 36)}

\newpage
%31----------------------------------------------------------------------
\item Foi realizada uma pesquisa de algumas características físicas entre a população de uma determinada região. Os dados coletados por habitante, para serem analisados foram:
\begin{itemize}
    \item sexo (masculino ou feminino);
    \item cor dos olhos (azuis, verdes ou castanhos);
    \item cor dos cabelos (louros, castanhos, pretos);
    \item idade.
\end{itemize}
Escrever um algoritmo que obtenha e escreva:
\begin{itemize}
    \item a maior idade entre os habitantes;
    \item a quantidade de indivíduos do sexo feminino cuja idade está entre 18 e 24 anos inclusive e que tenham olhos verdes e cabelos louros.
\end{itemize}
O final do algoritmo ocorrerá quando entrar uma idade negativa.\\
{\tiny Adaptado de: Orth (2001, p. 36-37)}

%32----------------------------------------------------------------------
\item Escrever um algoritmo que lê uma quantidade não determinada de números positivos. Calcula e escreve a quantidade de números pares e ímpares, a média dos valores pares e dos ímpares e a média geral dos valores lidos. O algoritmo finaliza quando entrar um valor nulo.\\
{\tiny Adaptado de: Orth (2001, p. 37)}

%33----------------------------------------------------------------------
\item Escrever um algoritmo que lê o código e o valor de cada produto de uma empresa. Supondo que a empresa deseja aumentar todos os produtos de códigos pares em 15\% e todos os produtos cujos códigos são ímpares em 20\%, escreva a lista dos produtos com o respectivo código e o seu preço já reajustado. O algoritmo termina quando entrar um código nulo.\\
{\tiny Adaptado de: Orth (2001, p. 37)}

%34----------------------------------------------------------------------
\item Escrever um algoritmo que gera os números entre 1000 e 2000 e escreve aquele que divididos por 11 deixam resto igual a 5.\\
{\tiny Adaptado de: Orth (2001, p. 37)}

%35----------------------------------------------------------------------
\item Escrever um algoritmo que lê 50 valores inteiros e positivos e:
\begin{enumerate}[a)]
    \item apresenta o maior dos 50 valores lidos;
    \item apresenta o menor dos 50 valores lidos;
    \item apresenta a média aritmética dos 50 valores lidos.
\end{enumerate}
{\tiny Adaptado de: Orth (2001, p. 37)}

%36----------------------------------------------------------------------
\item Escrever um algoritmo que lê um valor inteiro e positivo \texttt{n} e calcula a seguinte soma:
\[ S = 1 + \frac{1}{2} + \frac{1}{3} + \frac{1}{4} + ... + \frac{1}{n}\]
O algoritmo deve escrever cada um dos termos da soma e o valor final de \texttt{S}.\\
{\tiny Adaptado de: Orth (2001, p. 37)}

%37----------------------------------------------------------------------
\item Escrever um algoritmo que calcula e escreve a média ponderada com peso 1 para os valores ímpares e peso 2 para os valores pares situados no intervalo [13, 77].\\
{\tiny Adaptado de: Orth (2001, p. 37)}

%38----------------------------------------------------------------------
\item Escrever um algoritmo que lê 50 pares de valores, cada par formado pela altura e pelo sexo de uma pessoa (código=1, masculino; e código=2, feminino) e escreve a maior e menor altura para cada sexo.\\
{\tiny Adaptado de: Orth (2001, p. 37)}

%39----------------------------------------------------------------------
\item Escrever um algoritmo que calcula e escreve o produto dos números primos entre 95 e 1475.\\
{\tiny Adaptado de: Orth (2001, p. 37)}

\newpage
%40----------------------------------------------------------------------
\item Foi feita uma estatística nas 20 principais cidades do estado para coletar dados sobre acidentes de trânsito. Foram obtidos os seguintes dados:
\begin{itemize}
    \item código da cidade (1 a 20);
    \item número de veículos de passeio do ano passado;
    \item número de acidentes com vítimas do ano passado.
\end{itemize}
Escrever um algoritmo que lê 20 conjuntos de valores, um de cada vez, contendo o código da cidade, o número de veículos de passeio e o número de acidentes e, calcula e escreve:
\begin{enumerate}[a)]
    \item qual o maior e menor índice de acidentes e a que cidade pertencem respectivamente;
    \item qual a média de veículos nas cidades pesquisadas;
    \item qual a média de acidentes entre as 20 cidades pesquisadas.
\end{enumerate}
{\tiny Adaptado de: Orth (2001, p. 37-38)}

%41----------------------------------------------------------------------
\item Fez-se uma pesquisa entre os 1200 habitantes de uma região para coletar os seguintes dados: sexo (0 para feminino e 1 para masculino), idade e altura. Escrever um algoritmo que lê as informações coletadas e escreve as seguintes informações:
\begin{enumerate}[a)]
    \item média da idade do grupo;
    \item média da altura das mulheres;
    \item média da idade dos homens;
    \item percentual de pessoas com idade entre 21 e 53 anos.
\end{enumerate}
{\tiny Adaptado de: Orth (2001, p. 38)}

%42----------------------------------------------------------------------
\item Escrever um algoritmo que lê um conjunto não conhecido de pares de valores \texttt{a} e \texttt{b}, todos inteiros e positivos, e, para cada par de valores lido, obtém o M.D.C. de \texttt{a} e \texttt{b} e o escreve.\\
{\tiny Adaptado de: Orth (2001, p. 38)}

%43----------------------------------------------------------------------
\item Escrever um algoritmo que lê um conjunto não conhecido de pares de valores \texttt{a} e \texttt{b}, todos inteiros e positivos, e, para cada par de valores lido, obtém o M.D.C. e o M.M.C. de \texttt{a} e \texttt{b} e os escreve.\\
{\tiny Adaptado de: Orth (2001, p. 38)}

%44----------------------------------------------------------------------
\item Escrever um algoritmo que lê \texttt{n}, inteiro e positivo, e calcula e escreve o termo de ordem \texttt{n} da sucessão abaixo:
\begin{center}
\begin{tabular}{lrrrrrrrrrr}
\textbf{Ordem:} & ~ 1 & ~ 2 & ~ 3 & ~ 4 & ~ 5 & ~ 6 & ~ 7 & ~ 8 & ~ 9 & ~ 10 \\
\textbf{Sucessão:} & ~ -1 & ~ 0 & ~ 5 & ~ 6 & ~ 11 & ~ 12 & ~ 17 & ~ 18 & ~ 23 & ~ 24\\
\end{tabular}
\end{center}
{\tiny Adaptado de: Orth (2001, p. 38)}

%45----------------------------------------------------------------------
\item Foram entrevistadas 50 pessoas e coletados os seguintes dados:
\begin{enumerate}[a)]
    \item sexo: M (``masculino'') e F (``feminino'');
    \item cor dos olhos: A (``azuis''), V (``verdes''), e C (``castanhos'');
    \item cor dos cabelos: L (``louros''), C (``castanhos'') e P (``pretos'');
    \item idade.
\end{enumerate}
Escrever um algoritmo que lê os 50 conjuntos de valores para cada uma das pessoas entrevistadas e escreve:
\begin{enumerate}[a)]
    \item a maior idade do grupo;
    \item a quantidade de indivíduos do sexo feminino, com idade entre 18 e 24 anos e que tenham olhos verdes e cabelos louros.
\end{enumerate}
{\tiny Adaptado de: Orth (2001, p. 38)}

%46----------------------------------------------------------------------
\item Uma loja de departamentos que tem cadastrados 180 clientes deseja mandar uma correspondência aos melhores dos seus clientes anunciando-lhes um bônus especial. Escrever um algoritmo que lê o nome do cliente e o valor de suas compras no ano que passou e calcula um bônus de 15\% para aqueles clientes cujo valor de compras excedeu a 200 reais. Escrever o nome dos clientes que são merecedores deste brinde.\\
{\tiny Adaptado de: Orth (2001, p. 38)}

%47----------------------------------------------------------------------
\item Escreva um algoritmo que lê um conjunto não determinado de pares de datas e escreva a diferença entre elas em dias.\\
{\tiny Adaptado de: Orth (2001, p. 38)}

%48----------------------------------------------------------------------
\item Sabemos pela Lei de Newton que a força de atração entre dois corpos é diretamente proporcional ao produto de suas massas e inversamente proporcional ao quadrado da distância entre eles.
\[f = G \times \frac{m_1 \times m_2}{d^2}\]
Escrever um algoritmo que lê um número não conhecido de conjuntos de 4 valores \texttt{m1} (massa do primeiro corpo), \texttt{m2} (massa do segundo corpo), \texttt{G} (constante de gravitação universal) e \texttt{d} (distância entre os corpos) e calcula e escreve, para cada conjunto lido, a força de atração entre os dois corpos de massas \texttt{m1} e \texttt{m2}, respectivamente.
{\tiny Adaptado de: Orth (2001, p. 38-39)}

%49----------------------------------------------------------------------
\item Supondo que a população de uma cidade \textbf{A} seja da ordem de 90.000 habitantes com uma taxa anual de crescimento de 3,1\% e que a população de uma cidade \textbf{B} seja da ordem de 200.000 habitantes com uma taxa anual de crescimento de 1,5\%. Escrever um algoritmo que calcula quantos anos serão necessários para que a população da cidade \textbf{A} ultrapasse a população da cidade \textbf{B}, mantidas as taxas atuais de crescimento de cada uma.\\
{\tiny Adaptado de: Orth (2001, p. 39)}

%50----------------------------------------------------------------------
\item Supondo dados os três primeiros números primos, respectivamente 2, 3, e 5, escrever um algoritmo que escreve os 20 primeiros número primos, incluindo os três já dados.\\
{\tiny Adaptado de: Orth (2001, p. 39)}

\end{enumerate}

%----------------------------------------------------------------------
~\\
\noindent{\textbf{REFERÊNCIAS}}

\noindent{ORTH, Afonso Inácio. \textbf{Algoritmos e Programação com Resumo das Linguagens PASCAL e C}. Porto Alegre: AIO, 2001. 176 p.}

\end{document}

